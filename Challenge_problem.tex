\documentclass[journal,12pt,twocolumn]{IEEEtran}
%
\usepackage{setspace}
\usepackage{gensymb}
%\doublespacing
\singlespacing

%\usepackage{graphicx}
%\usepackage{amssymb}
%\usepackage{relsize}
\usepackage[cmex10]{amsmath}
%\usepackage{amsthm}
%\interdisplaylinepenalty=2500
%\savesymbol{iint}
%\usepackage{txfonts}
%\restoresymbol{TXF}{iint}
%\usepackage{wasysym}
\usepackage{amsthm}
%\usepackage{iithtlc}
\usepackage{mathrsfs}
\usepackage{txfonts}
\usepackage{stfloats}
\usepackage{mathtools}
\usepackage{bm}
\usepackage{cite}
\usepackage{cases}
\usepackage{subfig}
%\usepackage{xtab}
\usepackage{longtable}
\usepackage{multirow}
%\usepackage{algorithm}
%\usepackage{algpseudocode}
\usepackage{enumitem}
\usepackage{mathtools}
\usepackage{steinmetz}
\usepackage{tikz}
\usepackage{circuitikz}
\usepackage{verbatim}
\usepackage{tfrupee}
\usepackage[breaklinks=true]{hyperref}
%\usepackage{stmaryrd}
\usepackage{tkz-euclide} % loads  TikZ and tkz-base
%\usetkzobj{all}
\usetikzlibrary{calc,math}
\usepackage{listings}
    \usepackage{color}                                            %%
    \usepackage{array}                                            %%
    \usepackage{longtable}                                        %%
    \usepackage{calc}                                             %%
    \usepackage{multirow}                                         %%
    \usepackage{hhline}                                           %%
    \usepackage{ifthen}                                           %%
  %optionally (for landscape tables embedded in another document): %%
    \usepackage{lscape}     
\usepackage{multicol}
\usepackage{chngcntr}
%\usepackage{enumerate}

%\usepackage{wasysym}
%\newcounter{MYtempeqncnt}
\DeclareMathOperator*{\Res}{Res}
%\renewcommand{\baselinestretch}{2}
\renewcommand\thesection{\arabic{section}}
\renewcommand\thesubsection{\thesection.\arabic{subsection}}
\renewcommand\thesubsubsection{\thesubsection.\arabic{subsubsection}}

\renewcommand\thesectiondis{\arabic{section}}
\renewcommand\thesubsectiondis{\thesectiondis.\arabic{subsection}}
\renewcommand\thesubsubsectiondis{\thesubsectiondis.\arabic{subsubsection}}

% correct bad hyphenation here
\hyphenation{op-tical net-works semi-conduc-tor}
\def\inputGnumericTable{}                                 %%

\lstset{
%language=C,
frame=single, 
breaklines=true,
columns=fullflexible
}
%\lstset{
%language=tex,
%frame=single, 
%breaklines=true
%}

\begin{document}
%


\newtheorem{theorem}{Theorem}[section]
\newtheorem{problem}{Problem}
\newtheorem{proposition}{Proposition}[section]
\newtheorem{lemma}{Lemma}[section]
\newtheorem{corollary}[theorem]{Corollary}
\newtheorem{example}{Example}[section]
\newtheorem{definition}[problem]{Definition}
%\newtheorem{thm}{Theorem}[section] 
%\newtheorem{defn}[thm]{Definition}
%\newtheorem{algorithm}{Algorithm}[section]
%\newtheorem{cor}{Corollary}
\newcommand{\BEQA}{\begin{eqnarray}}
\newcommand{\EEQA}{\end{eqnarray}}
\newcommand{\define}{\stackrel{\triangle}{=}}

\bibliographystyle{IEEEtran}
%\bibliographystyle{ieeetr}


\providecommand{\mbf}{\mathbf}
\providecommand{\pr}[1]{\ensuremath{\Pr\left(#1\right)}}
\providecommand{\qfunc}[1]{\ensuremath{Q\left(#1\right)}}
\providecommand{\sbrak}[1]{\ensuremath{{}\left[#1\right]}}
\providecommand{\lsbrak}[1]{\ensuremath{{}\left[#1\right.}}
\providecommand{\rsbrak}[1]{\ensuremath{{}\left.#1\right]}}
\providecommand{\brak}[1]{\ensuremath{\left(#1\right)}}
\providecommand{\lbrak}[1]{\ensuremath{\left(#1\right.}}
\providecommand{\rbrak}[1]{\ensuremath{\left.#1\right)}}
\providecommand{\cbrak}[1]{\ensuremath{\left\{#1\right\}}}
\providecommand{\lcbrak}[1]{\ensuremath{\left\{#1\right.}}
\providecommand{\rcbrak}[1]{\ensuremath{\left.#1\right\}}}
\theoremstyle{remark}
\newtheorem{rem}{Remark}
\newcommand{\sgn}{\mathop{\mathrm{sgn}}}
\providecommand{\abs}[1]{\left\vert#1\right\vert}
\providecommand{\res}[1]{\Res\displaylimits_{#1}} 
\providecommand{\norm}[1]{\left\lVert#1\right\rVert}
%\providecommand{\norm}[1]{\lVert#1\rVert}
\providecommand{\mtx}[1]{\mathbf{#1}}
\providecommand{\mean}[1]{E\left[ #1 \right]}
\providecommand{\fourier}{\overset{\mathcal{F}}{ \rightleftharpoons}}
%\providecommand{\hilbert}{\overset{\mathcal{H}}{ \rightleftharpoons}}
\providecommand{\system}{\overset{\mathcal{H}}{ \longleftrightarrow}}
	%\newcommand{\solution}[2]{\textbf{Solution:}{#1}}
\newcommand{\solution}{\noindent \textbf{Solution: }}
\newcommand{\cosec}{\,\text{cosec}\,}
\providecommand{\dec}[2]{\ensuremath{\overset{#1}{\underset{#2}{\gtrless}}}}
\newcommand{\myvec}[1]{\ensuremath{\begin{pmatrix}#1\end{pmatrix}}}
\newcommand{\mydet}[1]{\ensuremath{\begin{vmatrix}#1\end{vmatrix}}}
%\numberwithin{equation}{section}
\numberwithin{equation}{subsection}
%\numberwithin{problem}{section}
%\numberwithin{definition}{section}
\makeatletter
\@addtoreset{figure}{problem}
\makeatother

\let\StandardTheFigure\thefigure
\let\vec\mathbf
%\renewcommand{\thefigure}{\theproblem.\arabic{figure}}
\renewcommand{\thefigure}{\theproblem}
%\setlist[enumerate,1]{before=\renewcommand\theequation{\theenumi.\arabic{equation}}
%\counterwithin{equation}{enumi}


%\renewcommand{\theequation}{\arabic{subsection}.\arabic{equation}}

\def\putbox#1#2#3{\makebox[0in][l]{\makebox[#1][l]{}\raisebox{\baselineskip}[0in][0in]{\raisebox{#2}[0in][0in]{#3}}}}
     \def\rightbox#1{\makebox[0in][r]{#1}}
     \def\centbox#1{\makebox[0in]{#1}}
     \def\topbox#1{\raisebox{-\baselineskip}[0in][0in]{#1}}
     \def\midbox#1{\raisebox{-0.5\baselineskip}[0in][0in]{#1}}

\vspace{3cm}


\title{Challenge Problem 3}
\author{Jayati Dutta}





% make the title area
\maketitle

\newpage

%\tableofcontents

\bigskip

\renewcommand{\thefigure}{\theenumi}
\renewcommand{\thetable}{\theenumi}
%\renewcommand{\theequation}{\theenumi}


%\begin{abstract}
%This is a simple document explaining how to determine the QR decomposition of a 2x2 matrix.
%\end{abstract}

%Download all python codes 
%
%\begin{lstlisting}
%svn co https://github.com/JayatiD93/trunk/My_solution_design/codes
%\end{lstlisting}

%Download all and latex-tikz codes from 
%%
%\begin{lstlisting}
%svn co https://github.com/gadepall/school/trunk/ncert/geometry/figs
%\end{lstlisting}
%%


\section{Problem}
Prove that - Complex-valued circulant matrices are simultaneously orthogonally diagonalizable.In other words, there exists an n×n matrix $P$ such that for every n×n circulant matrix $A$,the matrix $P^H AP$ is diagonal. Here, $P^H$ is the conjugate transpose/Hermitian (sometimes confusingly called the adjoint) of $P$. What are $P$ and $P^{-1}$?

\section{Explanation}
Let us consider $A$ be a nxn circulant matrix, that is,
\begin{align}
A = \myvec{C_0 & C_1 & C_2 & ...& C_{n-1}\\C_{n-1} & C_0 & C_1 & ...& C_{n-2}\\.& ...&..&..&.\\.& ...&..&..&.\\.& ...&..&..&.\\C_1 & .&..&...& C_0}
\end{align}
Where element of the matrix is considered as complex. So, considering the Hermitian transpose of the matrix $A$ we can get that:
\begin{align}
A^H = \myvec{C_0^* & C_1^* & C_2^* & ...& C_{n-1}^*\\C_{n-1}^* & C_0^* & C_1^* & ...& C_{n-2}^*\\.& ...&..&..&.\\.& ...&..&..&.\\.& ...&..&..&.\\C_1^* & .&..&...& C_0^*}^T\\
\implies A^H = \myvec{C_0^* & C_{n-1}^* & C_{n-2}^* & ...& C_1^*\\C_1^* & C_0^* & .. & ...& ..\\.& ...&..&..&.\\.& ...&..&..&.\\.& ...&..&..&.\\C_{n-1}^* & .&..&...& C_0^*}
\end{align}
It is observed that $A^H$ is also a circulant matrix.
One of the important properties of circulant matrices is that, all circulant matrices can commute to each other. So, matrix $A$ can commute to its Hermitian transpose matrix $A^H$, that is, 
\begin{align}
A A^H = A^H A
\end{align}
This implies that matrix $A$ is a normal matrix, that is why matrix $A$ has a full set of mutually orthogonal eigen vectors.
As all the circulant matrices can commute to each other, so they have same set of eigen vectors but different eigen values.

Let, $x^{(0)}$ is one eigen vector and
\begin{align}
x^{(0)} = \myvec{1\\1\\.\\.\\1}\\
\implies Ax^{(0)} = A \myvec{1\\1\\.\\.\\1}\\
\implies Ax^{(0)} = (C_0+C_1+C_2+....+C_{n-1})x^{(0)}\\
\implies Ax^{(0)} = \lambda_0 x^{(0)}
\end{align}

The eigen vectors can also be written as :
\begin{align}
\omega_n^{i} = e^{\frac{2\pi i}{n}}
\end{align}
So, the $k^{th}$ eigen vector of a circulant matrix is :
\begin{align}
x^{(k)} = \myvec{\omega_n^{0k}\\\omega_n^{1k}\\\omega_n^{2k}\\.\\.\\\omega_n^{(n-1)k}}
\end{align}
Now, let us consider a matrix $P$ whose columns are the eigen vectors, such that:
\begin{align}
P = \myvec{x^{(0)} & x^{(1)} &...&x^{(n-1)}}
\end{align}
These eigen vectors are mutually orthogonal.

If any vector is multiplied by the matrix $P$, then DFT operation can be performed.
The full form of DFT is Discrete Fourier Transform which is mainly used for numerical calculation in Digital Signal Processing. The DFT transforms N discrete-time samples to N discrete-frequency samples. If $x[n]$ is the discrete time samples and $X[k]$ is the discrete frequency samples,then the relation between them will be:
\begin{align}
X[k] = \sum_{n=0}^{N-1} x[n] \exp{(-\frac{2\pi n k}{N})}\\
\implies  X[k] = \sum_{n=0}^{N-1} x[n] \omega_{N}^{-nk}
\label{eqn1}
\end{align}
Where $N$ is the length of the sequence.Now, lets consider the length of the sequence is 2, that is, $N$ = 2. So $P$ will be :
\begin{align}
P = \myvec{x^{(0)} & x^{(1)}}
\end{align}
and 
\begin{align}
x^{(k)} = \myvec{\omega_n^{0k}\\\omega_n^{1k}}
\end{align}
So,
\begin{align}
x^{(0)} = \myvec{1\\1}\\
x^{(1)} = \myvec{1\\-1}
\end{align}
Now, 
\begin{align}
P = \myvec{1 & 1\\1 & -1}
\end{align}
Similarly, it can be also observed that when the length of sequence is 4x4,  the $P$ matrix is:
\begin{align}
P = \myvec{x^{(0)} & x^{(1)} & x^{(2)} & x^{(3)}}
\end{align}
and 
\begin{align}
x^{(k)} = \myvec{\omega_n^{0k}\\\omega_n^{1k}\\\omega_n^{2k}\\\omega_n^{3k}}
\end{align}
So, 
\begin{align}
P = \myvec{1 & 1 & 1 & 1\\1 & i & -1 & -i\\1 & -1 & 1 & -1\\1 & -i & -1 & i}
\end{align}
If we calculate $X[k]$ using \ref{eqn1}, we will get same value.
So $P$ is a DFT matrix and its columns are orthogonal. $P$ is symmetric and
\begin{align}
\frac{1}{n}P^H P=I\\
\implies \frac{1}{n}P^H = P^{-1}
\end{align}  
As $P$ is considered complex matrix .
Let's consider $F = \frac{1}{\sqrt{n}}P$, so 
\begin{align}
\frac{1}{\sqrt{n}}P^H \frac{1}{\sqrt{n}}P=I\\
\implies F^H F = I\\
\implies F^H = F^{-1}
\end{align}
So, $F$ is a unitary matrix, that implies that $P$ is also unitary matrix.
As the circulant matrix $A$ is normal, so it is unitarily diagonalizable, so
\begin{align}
D = P^{-1} A P\\
\implies D = \frac{1}{n}(P^H A P)
\end{align}
Now, for calculating eigen value let us consider an eigen vector $x^{(k)}$ such that
\begin{align}
Ax^{(k)}= y 
\end{align} 
Then the $l^{th}$ component is 
\begin{align}
y_l = \sum_{j=0}^{n-1} C_{j-l} \omega_n^{jk}\\
\implies y_l = \omega_n^{lk} \sum_{j=0}^{n-1}C_{j-l} \omega_n^{(j-l)k}
\end{align} 
But as $C_j$ and $\omega_n^j$ both are periodic, so 
\begin{align}
\sum_{j=0}^{n-1}C_{j-l} \omega_n^{(j-l)k} = \sum_{j=0}^{n-1} C_j \omega_n^{jk} = \alpha_k
\end{align} 
and $\omega_n^{lk}=x^{(k)}$
So,
\begin{align}
A x^{(k)} = \alpha_k x^{(k)}
\end{align} 
Where 
\begin{align}
\alpha_k = \sum_{j=0}^{n-1} C_j \omega_n^{jk}
\end{align} 
$\alpha_k$ is the $k^{th}$ eigen value. If $\alpha$ is a vector of all eigen values then $\alpha = \myvec{\alpha_0\\\alpha_1\\.\\.\\\alpha_{(n-1)}}$
So, $\alpha$ = $PC$ where $C$ is the first row of $A$ matrix, that is , the eigen values of $A$ are the DFT of the first row of the matrix $A$.
%\renewcommand{\theequation}{\theenumi}
%\begin{enumerate}[label=\thesection.\arabic*.,ref=\thesection.\theenumi]
%\numberwithin{equation}{enumi}
%\item Verification of the above problem using python code.\\
%\solution The  following Python code generates Fig. \ref{fig:parabola}
%\begin{lstlisting}
%codes/check_parab.py
%\end{lstlisting}
%
%So the solution is matching with the given plot, hence it is verified.
%%
%\end{enumerate}

\end{document}



